\documentclass[10pt]{wrceletter}

\name{Dennis Evangelista}
\position{Assistant Professor}
%\email{\href{mailto:evangeli@usna.edu}{\emph{evangeli@usna.edu}}}
\email{evangeli@usna.edu}
\telephone{410-293-6132}

\date{\today}

\usepackage{designature}
\signature{\vspace*{-0.7in}\includesignature\\Dennis Evangelista}
%\signature{Dennis Evangelista\\Assistant Professor} % title not needed if in letterhead
\address{\null} %{105 Maryland Avenue\\Annapolis, MD 21402} % leave blank, provided in letterhead
%\longindentation=0in % to change signature to be flushleft

\begin{document}
\begin{letter}{% recipient address here
Marshall Scholarship Selection Committee}

% opening here
\opening{}
\raggedright % if you like this
\setlength{\parindent}{15pt} % if you like this
I am happy to recommend MIDN 1/C Corwin Stites for the Marshall Scholarship.  Stites is an outstanding student and one of the most talented I have met at the United States Naval Academy; he would be an outstanding representative of America to the UK. I am currently his capstone advisor and Bowman project advisor at USNA. 

I was Mr.~Stite's instructor in EW202 (Principles of Mechatronics).  In EW202, Stites stood above all others with a 100\% on the final exam and an A overall in the class. He also successfully developed autonomous behaviors for a rover navigating a channel while avoiding magnets embedded beneath within it, using a suite of color, ultrasonic range, and Hall-effect sensors and a small ARM-based microcontroller. 

Mr.~Stites leads his senior engineering capstone team in the development of remote and autonomous temperature sensing devices for use in COVID-19 response. I am impressed by his ability to motivate the team, his ability to reach out to medical personnel and biologists to understand critical functional requirements, and his reliable ability to make the project work. 

In his Bowman project work, Mr.~Stites has shown great promise as a researcher and would be very successful in an engineering graduate school. Stites sought me out to help him develop a research proposal on undersea network architectures with great relevance to future naval capabilities. Stites' work is important for a future where manned submarines will be teamed with unmanned assets, where navies may have to operate (underwater; surface; and aerial) swarms in tight areas where connectivity, bandwidth, and logistics are concerns. The passion Stites has shown in learning about the background and setting up his research questions bodes well for his aptitude to contribute something exciting and relevant. 

In preparing his applications to Cambridge and Oxford, I understand Mr.~Stites is in contact with researchers who work on statistical signal processing and machine learning; estimation, search, and planning with an aim towards fundamental robotic problems. Sending Stites to the UK to work with top researchers on these topics would be of great benefit to the United States by adding research depth, international experience, scientific credibility, and exposure to allies' perspectives to the already impressive intellectual arsenal this man will bring to a future, networked, UUV-capable submarine force. 

\closing{~} % provides empty 

%\ps{post script here}
%\encl{enclosure here}
\end{letter}
\end{document}

%I read the letter and I think it looks pretty good but as you said, I think if you added some info about what I particularly want to do in terms of research and study in the UK it would carry some more weight (more info below). 
%Here's a focus of study/research and my top scholarship of interest for Cambridge: Thomas Pownall Scholarship to University of Cambridge (MPhil in Machine Learning and Machine Intelligence). I would be interested in working with Prof. Richard Turner (http://www.eng.cam.ac.uk/profiles/ret26). He does work with statistical signal processing and machine learning. Seems like this would possibly be a good area of research to look into better methods of processing acoustic signals and writing algorithms to identify submarine acoustic signatures better?
%Here's a focus of study/research and my top scholarship of interest for Oxford:  Karl F. Beyer Scholarship to Oxford University (Brasenose College, MS in Information, Vision, and Control Engineering). I would be interested in working with Professor Jonathan Gammell (https://ori.ox.ac.uk/ori-people/jon-gammell/). He leads research at the Oxford Robotics Institute and leads the Estimation, Search, and Planning (ESP) research group which seeks to develop understanding of fundamental robotic problems. This seems like a good place to do research on the design of UUVs that could be used in conjunction with submarines?
%I would say that the letter is on target for the program. The only thing I would consider adding is maybe a point about how you think research that I could do as a grad student in the UK would be very beneficial for the Submarine force, the Navy, and even for society and engineering as a whole. I would want to do something related to submarine research as a grad student so it wouldn't be too tough of a sell to say something like that I think. 
%In terms of addressing the letter, all I know is that it will be read by a panel of USNA faculty and Alumni so "UKISP Scholarship Selection Committee' should suffice?
%I'm going to ask you to also give me a rec for the Marshall Scholarship as well (if that's OK). The Marshall Scholarship is very similar to the UKISP program. It's for studying at a UK grad school so you could reuse the same points from the UKISP letter. The only difference is that the Marshall Scholarship letter will be reviewed by a non-USNA panel so you might have to clarify any USNA-centric points. Other than that the letter for the UKISP program can be pretty easily repurposed for the Marshall.  Due date for submitting the Marshall is 15 SEP. All the instructions for submitting these things are in the documents I sent you (attached again for convenience).
%Sorry for the late response and the late email. We can talk more about this in our next meeting which I will reschedule with you in a different email. 
