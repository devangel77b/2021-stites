\documentclass[10pt]{wrceletter}

\name{Dennis Evangelista}
\position{Assistant Professor}
%\email{\href{mailto:evangeli@usna.edu}{\emph{evangeli@usna.edu}}}
\email{evangeli@usna.edu}
\telephone{410-293-6132}

\date{\today}

\usepackage{designature}
\signature{\vspace*{-0.7in}\includesignature\\Dennis Evangelista}
%\signature{Dennis Evangelista\\Assistant Professor} % title not needed if in letterhead
\address{\null} %{105 Maryland Avenue\\Annapolis, MD 21402} % leave blank, provided in letterhead
%\longindentation=0in % to change signature to be flushleft

\begin{document}
\begin{letter}{% recipient address here
California Institute of Technology\\
Graduate Admissions\\
1200 East California Boulevard\\
Pasadena, CA 91125}

% opening here
\opening{}
\raggedright % if you like this
\setlength{\parindent}{15pt} % if you like this
I am happy to recommend MIDN 1/C Corwin Stites for graduate school at Caltech.  Stites is an outstanding student and one of the most talented I have met at the United States Naval Academy. I am currently his senior capstone project advisor at USNA, and also advise Stites as a Frank L.~Bowman Scholar. 

I was Mr.~Stites' instructor in EW202 (Principles of Mechatronics).  In EW202, Stites stood above all others with a 100\% on the final exam and an A overall in the class. He also successfully developed autonomous behaviors for a rover navigating a channel while avoiding magnets simulating mines, using a suite of color, ultrasonic range, and Hall-effect sensors and a small ARM-based microcontroller. 

Mr.~Stites leads his senior engineering capstone team in the development of remote and autonomous vital sign and temperature sensing devices for use in COVID-19 response. I am impressed by his ability to motivate the team and his ability to reach out to medical personnel and biologists to understand critical functional requirements. At a time when much of DOD and the federal government appears in denial, Stites led his team of engineers to consider this project in order to benefit others and apply engineering skills to pandemic response. In doing so, Stites demonstrates independence of thought, purposeful leadership, and civic mindset; in USNA terms he is reaching for higher responsibilities in command, citizenship, and government.

The Admiral Frank L.~Bowman Scholar Program provides for immediate graduate education for the top midshipmen who have been selected for naval nuclear propulsion. In his Bowman project work, Mr.~Stites has shown great promise as a researcher and would be very successful in an engineering graduate school. Stites sought me out to help him develop a research proposal on undersea network architectures with great relevance to future naval capabilities. In this area, Stites' work is important for a future where manned submarines will be teamed with unmanned assets, where navies may have to operate (underwater; surface; and aerial) swarms in tight areas where connectivity, bandwidth, and logistics are concerns. Much of USNA teaching and advising is undergraduate-focused, but on this project I treat Stites like a graduate student encouraging independence of thought and making him own his project. The passion Stites has shown in learning about the background and setting up his research questions bodes well for his aptitude to contribute something exciting and relevant. My constructive criticism of Stites is that USNA can be too controlling of an environment for further growth -- with added opportunities to exercise the intellectual freedom and autonomy of a graduate student in pursuing topics of interest, Stites will excel even more, especially in ares of mobile networking research.

I have advised undergraduates at MIT, Berkeley, and UNC in addition to USNA; in areas at the intersection of biology and engineering that require navigating ambiguity, unfamiliar new fields, and may require confronting or adopting dissenting points of view. Stites ranks well in all of these. I served in the Navy as well, and Stites' Navy experience, in a US military experiencing rapid change and challenges, demonstrates his purposeful leadership and civic mindset. He is an outstanding candidate and will do well at Caltech.

\closing{~} % provides empty 

%\ps{post script here}
%\encl{enclosure here}
\end{letter}
\end{document}

%Paragraph 1, change NROTC program to Electrical Engineering (MS). Paragraph 2, sentence 3, change "beneath within it" to "beneath it." Paragraph 3, sentence 1, change "in the development of remote and vital sign and autonomous temperature sensing devices" to "in the development of remote and autonomous vital sign and temperature sensing devices"
%[9:12 AM]
%If you could throw a point in there about how my research work could carry over well into Stanford's EE Department's Mobile Networking research, that might be good. Mobile Networking: https://ee.stanford.edu/research/mobile-networking




%We ask your recommenders to assess some character and leadership traits in the reference letters, and to address the following topics:
%
%Please explain how you know and interact with the applicant.
%How have the applicant’s actions and insights benefited others? Please provide specific examples.
%What is the most important instance of constructive feedback that you have given the applicant? Please explain the circumstances and how the applicant responded?
%What does the candidate need to continue her/his/their intellectual development and professional growth?
%Is there anything else — positive or negative — that we should know about the applicant?
%[1:36 PM]
%What the Knight Hennessey people are looking for:
%[1:36 PM]
%1. Independence of Thought
%
%Exhibits first-step mental sharpness
%Seeks out knowledge and new experiences
%Full of original ideas
%Makes sense of ambiguity
%Can hold a contrarian or dissenting point of view
%2. Purposeful Leadership
%
%Ambitious, in the best sense
%Driven to improve self
%Able to bring others along
%Self-aware
%Persistent and resilient
%3. Civic Mindset
%
%Personally humble and kind
%Supportive and accountable
%Embraces difference
%Concerned for and helpful to others
%[1:37 PM]
%Hope this helps to give the letter some direction!
%[1:38 PM]
%Knight Hennesy link: https://knight-hennessy.stanford.edu/
%
