\documentclass[10pt]{wrceletter}

\name{Dennis Evangelista}
\position{Assistant Professor}
%\email{\href{mailto:evangeli@usna.edu}{\emph{evangeli@usna.edu}}}
\email{evangeli@usna.edu}
\telephone{410-293-6132}

\date{\today}

\usepackage{designature}
\signature{\vspace*{-0.7in}\includesignature\\Dennis Evangelista}
%\signature{Dennis Evangelista\\Assistant Professor} % title not needed if in letterhead
\address{\null} %{105 Maryland Avenue\\Annapolis, MD 21402} % leave blank, provided in letterhead
%\longindentation=0in % to change signature to be flushleft

\begin{document}
\begin{letter}{% recipient address here
Selection Committee\\
National Defense Science and Engineering Graduate Fellowship}

% opening here
\opening{}
\raggedright % if you like this
\setlength{\parindent}{15pt} % if you like this
I am happy to recommend MIDN 1/C Corwin Stites for the NSDSEG program. I was one of Stites' instructors and was previously a Naval Reactors engineer, where I worked on advanced submarine systems and deep submergence. Stites is an outstanding student and one of the most talented I have met at the United States Naval Academy. I am currently his senior capstone project advisor at USNA, and also advise Stites as a Frank L.~Bowman Scholar.  Stites has great potential for graduate study, is motivated to serve in submarines, and is passionate about autonomy and distributed sensing topics of timely relevance to national defense.  

I was Mr.~Stites' instructor in EW202 (Principles of Mechatronics).  In EW202, Stites stood above all others with a 100\% on the final exam and an A overall in the class. He also successfully developed autonomous behaviors for a rover navigating a channel while avoiding magnets embedded beneath within it, using a suite of color, ultrasonic range, and Hall-effect sensors and a small ARM-based microcontroller. 

Mr.~Stites leads his senior engineering capstone team in the development of remote and vital sign and autonomous temperature sensing devices for use in COVID-19 response. I am impressed by his ability to motivate the team and his ability to reach out to medical personnel and biologists to understand critical functional requirements. Stites led his team of engineers to consider this project in order to benefit others and apply engineering skills to pandemic response. In doing so, Stites demonstrates independence of thought, purposeful leadership, and civic mindset; in USNA terms he is reaching for higher responsibilities in command, citizenship, and government.

In graduate school, Mr.~Stites wishes to study networks, autonomy, and distributed control for UUVs. I believe he will be very succesful in this field. He is currently a Bowman Scholar working in undersea autonomous warfare; a future where manned submarines will be teamed with unmanned assets, and we as well as our opponents may have to operate (underwater; surface; and aerial) swarms in tight areas where connectivity, bandwidth, and logistics are concerns. The passion Stites has shown in developing and answering deep research questions bodes well for his aptitude to contribute something exciting and relevant. Stites' Bowman project makes good use of modeling and simulation and the unique features of the new Hopper Hall for UUV studies. It also connects with summer work done at internships, and is of interest to the nuclear Navy.  

NDSEG would allow Stites to continue on this path, following up on deeper questions than he is able to answer in his final year at USNA and exposing him to a broader range of perspectives encountered in a graduate school environment. He is the finest candidate you could consider for your program. 

\closing{~} % provides empty 

%\ps{post script here}
%\encl{enclosure here}
\end{letter}
\end{document}

